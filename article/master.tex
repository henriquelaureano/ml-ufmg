\documentclass[ignorenonframetext,]{beamer}
\setbeamertemplate{caption}[numbered]
\setbeamertemplate{caption label separator}{: }
\setbeamercolor{caption name}{fg=normal text.fg}
\beamertemplatenavigationsymbolsempty
\usepackage{lmodern}
\usepackage{amssymb,amsmath}
\usepackage{ifxetex,ifluatex}
\usepackage{fixltx2e} % provides \textsubscript
\ifnum 0\ifxetex 1\fi\ifluatex 1\fi=0 % if pdftex
\usepackage[T1]{fontenc}
\usepackage[utf8]{inputenc}
\else % if luatex or xelatex
\ifxetex
\usepackage{mathspec}
\else
\usepackage{fontspec}
\fi
\defaultfontfeatures{Ligatures=TeX,Scale=MatchLowercase}
\fi
\usetheme{Szeged}
\usecolortheme{rose}
\usefonttheme{structurebold}
% use upquote if available, for straight quotes in verbatim environments
\IfFileExists{upquote.sty}{\usepackage{upquote}}{}
% use microtype if available
\IfFileExists{microtype.sty}{%
\usepackage{microtype}
\UseMicrotypeSet[protrusion]{basicmath} % disable protrusion for tt fonts
}{}
\newif\ifbibliography

% Prevent slide breaks in the middle of a paragraph:
\widowpenalties 1 10000
\raggedbottom

\AtBeginPart{
\let\insertpartnumber\relax
\let\partname\relax
\frame{\partpage}
}
\AtBeginSection{
\ifbibliography
\else
\let\insertsectionnumber\relax
\let\sectionname\relax
\frame{\sectionpage}
\fi
}
\AtBeginSubsection{
\let\insertsubsectionnumber\relax
\let\subsectionname\relax
\frame{\subsectionpage}
}

\setlength{\parindent}{0pt}
\setlength{\parskip}{6pt plus 2pt minus 1pt}
\setlength{\emergencystretch}{3em}  % prevent overfull lines
\providecommand{\tightlist}{%
\setlength{\itemsep}{0pt}\setlength{\parskip}{0pt}}
\setcounter{secnumdepth}{0}
\institute[est171]{est171 - Aprendizado de Máquina \\ Pós-Graduação em Estatística \\ UFMG - Universidade Federal de Minas Gerais}
\setbeamertemplate{navigation symbols}{}
\usepackage{bm}
\usepackage{multicol}

\title{Applications of continuous time hidden Markov models to the study of
misclassified disease outcomes}
\author{Henrique Aparecido Laureano}
\date{21 de novembro de 2016}

\begin{document}
\frame{\titlepage}

\begin{frame}{Roteiro}

\tableofcontents

\end{frame}

\section{Contextualizando}\label{contextualizando}

\begin{frame}

\vspace{.25cm}\begin{block}{Autores}
 Alexander Bureau \\
 \hskip 3cm \textit{Group in Biostatistics, School of Public Health, \\
 \hskip 3cm         University of California, Berkeley} \\
 \vspace{.25cm}
 Stephen Shiboski \\
 \hskip 3cm \textit{Department of Epidemiology and Biostatistics, \\
 \hskip 3cm         University of California, San Francisco} \\
 \vspace{.25cm}
 James P. Hughes \\
 \hskip 2.5cm \textit{Department of Biostatistics, \\
 \hskip 2.5cm         School of Public Health and Community Medicine, \\
 \hskip 2.5cm         University of Washington}
\end{block}

\end{frame}

\begin{frame}

\vspace{.2cm}\begin{columns}
 \begin{column}{.33\textwidth}
  \begin{center}
   \includegraphics*[height = 4.5cm, width = 3.5cm]{bureau.jpg}
  \end{center}
 \end{column}
 \begin{column}{.33\textwidth}
  \begin{center}
    \includegraphics*[height = 4.5cm, width = 3.5cm]{shiboski.jpg}
  \end{center}
 \end{column}
 \begin{column}{.33\textwidth}
  \begin{center}
    \includegraphics*[height = 4.5cm, width = 3.5cm]{hughes.jpg}
  \end{center}
 \end{column}
\end{columns}

\vspace{.3cm}\begin{block}{Publicação}
 \textit{Statistics in Medicine}. 2003; \textbf{22}:441-462
 (DOI: 10.1002/sim.1270)
\end{block}

\end{frame}

\begin{frame}

\vspace{.25cm}\begin{block}{Artigo}
 Aplicação de modelos markovianos ocultos de tempo contínuo em medidas
 longitudinais de doenças com resposta binária \\
 \vspace{.25cm}
 A resposta da doença é representada por um processo markoviano 
 homogêneo de dois estados em tempo contínuo
\end{block}

\vspace{.5cm}

\pause

\begin{block}{Modelo markoviano oculto}
 Extensão de modelos markovianos que fornece uma maneira de lidar com 
 possíveis erros de classificação devido ao processo de avaliação
\end{block}

\end{frame}

\section{Modelo markoviano oculto}\label{modelo-markoviano-oculto}

\begin{frame}

\vspace{.2cm}\begin{block}{Variáveis}
 \(T_{ij} \rightarrow
 \text{tempo cronológico das avaliações}\) \\
 \vspace{.2cm}
 \(Y_{ij} \rightarrow
 \text{medida dicotômica da resposta da doença}\) \\
 \vspace{.2cm}
 \(\textbf{Z}_{ij} \rightarrow
 \text{vetor } 1 \times p \text{ de covariáveis}\) \\
 \vspace{.3cm}
 em que \(i\) representa o indivíduo e \(j\) a avaliação \\
 \vspace{.3cm}
 \pause
 \(X(t), t > 0 \rightarrow \text{status atual subjacente da doença}\) \\
 \hskip 2.4cm (Modelado como um processo markoviano em tempo \\
 \hskip 2.6cm  contínuo de dois estados) \\
 \vspace{.3cm}
 \pause
 \(Y_{^1}^{j} \text{ e } T_{1}^{j}\) denotam a sequência de 1 até \(j\)
 de estados observados da doença e tempos de avaliação \\
 (O índice \(i\) denotando o indivíduo é omitido)
\end{block}

\end{frame}

\begin{frame}

\vspace{.3cm}

\pause

\begin{block}{Suposição markoviana para o processo oculto da doença}
 \vspace{-.45cm}
 \footnotesize
 \[\begin{aligned}
   P[X(t_{j})|X(t_{1}), ..., X(t_{j-1}), Y_{1}^{j-1},
   T_{1}^{j} = t_{1}^{j}] & = P[X(t_{j})|X(t_{j-1}),
                              T_{j-1}^{j} = t_{j-1}^{j}] \\
                          & = P_{x_{j-1}, x_{j}}(t_{j} - t_{j-1})
   \end{aligned}\]
\end{block}

\vspace{.6cm}

\pause

\begin{block}{Probabilidade de classificar corretamente ou
              incorretamente o estado da doença dado o verdadeiro
              estado}
 \vspace{-.45cm}
 \footnotesize
 \[P[Y_{j}|X(t_{1}), ..., X(t_{j}), Y_{1}^{j-1}, T_{1}^{j} = t_{1}^{j}]
   = P[Y_{j}|X(t_{j}), T_{j} = t_{j}] = f(y_{j}|x_{j})\]
\end{block}

\end{frame}

\begin{frame}

\vspace{.35cm}\begin{block}{Processo oculto em tempo contínuo de dois estados}
 \vspace{-.25cm}
 \begin{itemize}
  \pause \item Intensidade de aquisição da manifestação da doença, \(u\)
  \pause \item Intensidade de remoção, \(v\)
 \end{itemize}
\end{block}

\vspace{.5cm}

\pause

\begin{block}{Probabilidades de transição}
 \vspace{-.5cm}
 \small
 \[P_{01}(t) = \frac{u}{u+v}(1-\exp\{-(u+v)t\}) \quad
   P_{10}(t) = \frac{v}{u+v}(1-\exp\{-(u+v)t\})\]
\end{block}

\end{frame}

\begin{frame}

\vspace{.15cm}\begin{block}{Intensidade de aquisição sobre o intervalo
              [\(t_{j-1}, t_{j}\)] como função de \(p\) covariáveis}
 \vspace{-.5cm}
 \[u_{i, j}(z_{ij}) =
   \exp\left\{\theta_{u0}+\sum_{k=1}^{p}\theta_{uk}z_{ijk}\right\},\] \\
 \vspace{.15cm}
 \pause a expressão para a intensidade de remoção é similar
\end{block}

\vspace{-.1cm}

\pause

\begin{block}{Verossimilhança para as \(m\) observações de um indivíduo}
 \vspace{-.95cm}
 \pause
 \[\begin{aligned}
   L(\theta) & = P[Y_{1}^{m}|T_{1}^{m} = t_{1}^{m},
                             Z_{1}^{m} = z_{1}^{m}, \theta] \\
             & = \sum_{x_{1} ... x_{m}}
                 P[Y_{1}^{m}, X_{1}^{m}|T_{1}^{m} = t_{1}^{m},
                                        Z_{1}^{m} = z_{1}^{m}, \theta]\\
             & = \sum_{x_{1} ... x_{m}}
                 \pi_{x_{1}|z_{1}} f(y_{1}|x_{1}, z_{1})
                 \prod_{j = 2}^{m}
                  P_{x_{j-1}, x_{j}}(t_{j} - t_{j-1})
                  f(y_{j}|x_{j}, z_{j})
   \end{aligned}\]
 \vspace{-.5cm}
\end{block}

\end{frame}

\section{Aplicações}\label{aplicacoes}

\subsection{Leucoplasia pilosa}\label{leucoplasia-pilosa}

\begin{frame}

\vspace{.25cm}

\pause

\begin{block}{Leucoplasia pilosa}
 \pause
 \vspace{-.2cm}
 Lesão na língua causada pelo Vírus Epstein-Barr (VEB), acredita-se \\
 \vspace{.2cm}
 Muito frequente em pacientes infectados pelo vírus HIV
\end{block}

\vspace{.1cm}\begin{columns}
 \begin{column}{.4\textwidth}

\pause
  \begin{block}{Estudo}
   334 homens do \textit{San Francisco Men's Health Study}
  \pause
   \begin{itemize}
    \item Duração: fevereiro de 1987 até maio 1993
    \item Avaliação: Semestral
   \end{itemize}
  \end{block}
 
 \end{column}
 \begin{column}{.6\textwidth}
 \vspace{.25cm}
 \pause
  \includegraphics*[
   height = 4.75cm, width = 6.75cm]{iBagens/unnamed-chunk-1-1.png}

 \end{column}
\end{columns}

\end{frame}

\begin{frame}

\footnotesize \textit{Intensidades de transição (transição/mês)
                      estimadas para um modelo markoviano e para o
                      modelo markoviano oculto}

\includegraphics*[height = 1.7cm]{iBagens/t1.png}

\pause
\vspace{.2cm}
\textit{Estimativas dos efeitos das covariáveis em modelos com apenas  
        uma covariável}

\includegraphics*[height = 3.1cm]{iBagens/t2.png}

\end{frame}

\subsection{Infecção por HPV}\label{infeccao-por-hpv}

\begin{frame}

\pause

\begin{block}{Infecção por HPV}
 \pause
 \vspace{-.2cm}
 O HPV (papilomavírus humano) compreende uma diversidade grande de
 subtipos (mais de 70), que provocam desde o aparecimento de verrugas na
 pele e nas mucosas até doenças graves como o câncer do colo do útero \\
 \vspace{.4cm}
 Os subtipos 16 e 18 são de alto risco para o câncer do colo do útero
\end{block}

\end{frame}

\begin{frame}

\begin{columns}
 \begin{column}{.4\textwidth}

  \begin{block}{Estudo}
  \pause
   663 mulheres em San Francisco
  \pause
   \begin{itemize}
    \item Início: 1990
    \item Avaliação: Quadrimestral
   \end{itemize}
  \end{block}
 
 \end{column}
 \begin{column}{.6\textwidth}
 \vspace{.25cm}
 \pause
  \includegraphics*[
   height = 5cm, width = 6.75cm]{iBagens/unnamed-chunk-1-2.png}

 \end{column}
\end{columns}

\end{frame}

\begin{frame}

\textit{Itensidades de transição (transição/mês) estimadas para o 
        modelo markoviano oculto}

\includegraphics*[height = 1.6cm]{iBagens/t3.png}

\vspace{.5cm}

\pause
\textit{Probabilidades de observação dado os estados ocultos}

\centering \includegraphics*[height = 1.7cm]{iBagens/t4.png}

\end{frame}

\section{Considerações finais}\label{consideracoes-finais}

\begin{frame}

\begin{block}{Considerações finais}
 \pause
 Extensões interessantes a serem desenvolvidas:
 \begin{itemize}
  \item Possibilitar a interação entre múltiplos processos ocultos
 \end{itemize}
 \pause
 Modificações para possibilitar que intensidades de transições variem ao
 longo do tempo \\
 \vspace{.25cm}
 \pause
 Elaboração de técnicas de diagnóstico mais sofisticadas
\end{block}

\end{frame}

\begin{frame}

\LARGE \centering Obrigado por seu tempo!

\end{frame}

\end{document}
