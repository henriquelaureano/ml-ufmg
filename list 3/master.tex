\documentclass[12pt]{article}\usepackage[]{graphicx}\usepackage[]{color}
%% maxwidth is the original width if it is less than linewidth
%% otherwise use linewidth (to make sure the graphics do not exceed the margin)
\makeatletter
\def\maxwidth{ %
  \ifdim\Gin@nat@width>\linewidth
    \linewidth
  \else
    \Gin@nat@width
  \fi
}
\makeatother

\definecolor{fgcolor}{rgb}{0, 0, 0}
\newcommand{\hlnum}[1]{\textcolor[rgb]{0.502,0,0.502}{\textbf{#1}}}%
\newcommand{\hlstr}[1]{\textcolor[rgb]{0.651,0.522,0}{#1}}%
\newcommand{\hlcom}[1]{\textcolor[rgb]{1,0.502,0}{#1}}%
\newcommand{\hlopt}[1]{\textcolor[rgb]{1,0,0.502}{\textbf{#1}}}%
\newcommand{\hlstd}[1]{\textcolor[rgb]{0,0,0}{#1}}%
\newcommand{\hlkwa}[1]{\textcolor[rgb]{0.733,0.475,0.467}{\textbf{#1}}}%
\newcommand{\hlkwb}[1]{\textcolor[rgb]{0.502,0.502,0.753}{\textbf{#1}}}%
\newcommand{\hlkwc}[1]{\textcolor[rgb]{0,0.502,0.753}{#1}}%
\newcommand{\hlkwd}[1]{\textcolor[rgb]{0,0.267,0.4}{#1}}%
\let\hlipl\hlkwb

\usepackage{framed}
\makeatletter
\newenvironment{kframe}{%
 \def\at@end@of@kframe{}%
 \ifinner\ifhmode%
  \def\at@end@of@kframe{\end{minipage}}%
  \begin{minipage}{\columnwidth}%
 \fi\fi%
 \def\FrameCommand##1{\hskip\@totalleftmargin \hskip-\fboxsep
 \colorbox{shadecolor}{##1}\hskip-\fboxsep
     % There is no \\@totalrightmargin, so:
     \hskip-\linewidth \hskip-\@totalleftmargin \hskip\columnwidth}%
 \MakeFramed {\advance\hsize-\width
   \@totalleftmargin\z@ \linewidth\hsize
   \@setminipage}}%
 {\par\unskip\endMakeFramed%
 \at@end@of@kframe}
\makeatother

\definecolor{shadecolor}{rgb}{.97, .97, .97}
\definecolor{messagecolor}{rgb}{0, 0, 0}
\definecolor{warningcolor}{rgb}{1, 0, 1}
\definecolor{errorcolor}{rgb}{1, 0, 0}
\newenvironment{knitrout}{}{} % an empty environment to be redefined in TeX

\usepackage{alltt}
\usepackage[brazilian, brazil]{babel}
\usepackage[latin1]{inputenc}
\usepackage[T1]{fontenc}
\usepackage{amsmath}
\usepackage{graphicx}
\usepackage[top = 2.5cm, left = 2.5cm, right = 2.5cm, bottom = 2.5cm]{geometry}
\usepackage{indentfirst}
\usepackage{float}
\usepackage{multicol}
\usepackage[normalem]{ulem}
\usepackage{breqn}
\usepackage{amsfonts}
\usepackage{amsthm}
\usepackage{enumitem}
\usepackage{booktabs}
\setlength\parindent{0pt}
\newcommand{\eqnb}{\begin{equation}}
\newcommand{\eqne}{\end{equation}}
\newcommand{\eqnbs}{\begin{equation*}}
\newcommand{\eqnes}{\end{equation*}}
\newcommand{\horrule}[1]{\rule{\linewidth}{#1}}

\title{  
 \normalfont \normalsize 
 \textsc{est171 - Aprendizado de M�quina} \\
 Departamento de Estat�stica \\
 Universidade Federal de Minas Gerais \\ [25pt]
 \horrule{.5pt} \\ [.4cm]
 \huge Lista  3 \\
 \horrule{2pt} \\[ .5cm]}
 
\author{Henrique Aparecido Laureano \and Magno Tairone de Freitas Severino}
\date{\normalsize Outubro de 2016}
\IfFileExists{upquote.sty}{\usepackage{upquote}}{}
\begin{document}

\maketitle

\vspace{\fill}

\tableofcontents

\horrule{1pt} \\

\newpage



\section*{Exerc�cio I}
\addcontentsline{toc}{section}{Exerc�cio I}

\horrule{1pt} \\

\textbf{Seu objetivo � criar classificadores para predizer o d�gito
        correspondente a uma imagem. O conjunto de dados est� dispon�vel em
        https://www.kaggle.com/c/digit-recognizer.Voc� deve usar o arquivo
        train.csv para criar os seus classificadores (incluindo valida��o), e
        deve fornecer ao site as predi��es encontradas para o conjunto
        test.csv. Note que os d�gitos correspondentes no conjunto teste n�o
        est�o indicados! O site ir� rankear seu grupo de acordo com as
        predi��es fornecidas. Como parte do exerc�cio, voc� dever�:}

\begin{knitrout}\small
\definecolor{shadecolor}{rgb}{1, 1, 1}\color{fgcolor}\begin{kframe}
\begin{alltt}
\hlcom{# <code r> ================================================================== #}
\hlstd{path} \hlkwb{<-} \hlstr{"C:/Users/henri/Dropbox/Scripts/aprendizado de maquina/list 3/"}

\hlstd{train} \hlkwb{<-} \hlkwd{read.csv2}\hlstd{(}\hlkwd{paste0}\hlstd{(path,} \hlstr{"train.csv"}\hlstd{)}
                   \hlstd{,} \hlkwc{header} \hlstd{=} \hlnum{TRUE}
                   \hlstd{,} \hlkwc{sep} \hlstd{=} \hlstr{","}\hlstd{) ;} \hlkwd{dim}\hlstd{(train)}
\hlcom{# </code r> ================================================================= #}
\end{alltt}
\begin{verbatim}
[1] 42000   785
\end{verbatim}
\end{kframe}
\end{knitrout}

\textbf{a) Inscrever seu time Kaggle. Qual o nome dele?}

\horrule{.5pt} \\

\textbf{b) Plotar 5 imagens deste banco.}

\horrule{.5pt}

\begin{knitrout}\small
\definecolor{shadecolor}{rgb}{1, 1, 1}\color{fgcolor}\begin{kframe}
\begin{alltt}
\hlcom{# <code r> ================================================================== #}
\hlkwd{library}\hlstd{(jpeg)}

\hlstd{view} \hlkwb{<-} \hlkwa{function}\hlstd{(}\hlkwc{row}\hlstd{)\{}
  \hlstd{c} \hlkwb{<-} \hlnum{1}
  \hlstd{m} \hlkwb{<-} \hlkwd{matrix}\hlstd{(}\hlnum{NA}\hlstd{,} \hlnum{28}\hlstd{,} \hlnum{28}\hlstd{)}
  \hlkwa{for} \hlstd{(i} \hlkwa{in} \hlnum{1}\hlopt{:}\hlnum{28}\hlstd{)\{}
    \hlkwa{for} \hlstd{(j} \hlkwa{in} \hlnum{1}\hlopt{:}\hlnum{28}\hlstd{)\{}
      \hlstd{m[i, j]} \hlkwb{<-} \hlstd{train[row,} \hlnum{1} \hlopt{+} \hlstd{c]}
      \hlstd{c} \hlkwb{<-} \hlstd{c} \hlopt{+} \hlnum{1}\hlstd{\}\}}
  \hlkwd{return}\hlstd{(}\hlkwd{t}\hlstd{(}\hlkwd{apply}\hlstd{(m,} \hlnum{2}\hlstd{, rev)))\}}

\hlkwd{par}\hlstd{(}\hlkwc{mfrow} \hlstd{=} \hlkwd{c}\hlstd{(}\hlnum{2}\hlstd{,} \hlnum{3}\hlstd{),} \hlkwc{mar} \hlstd{=} \hlkwd{c}\hlstd{(}\hlnum{2}\hlstd{,} \hlnum{2}\hlstd{,} \hlnum{2}\hlstd{,} \hlnum{2}\hlstd{)} \hlopt{+} \hlnum{.1}\hlstd{)}

\hlkwa{for} \hlstd{(i} \hlkwa{in} \hlnum{1}\hlopt{:}\hlnum{5}\hlstd{)} \hlkwd{image}\hlstd{(}\hlkwd{view}\hlstd{(i),} \hlkwc{col} \hlstd{=} \hlkwd{grey}\hlstd{(}\hlkwd{c}\hlstd{(}\hlnum{0}\hlopt{:}\hlnum{255}\hlstd{)}\hlopt{/}\hlnum{256}\hlstd{),} \hlkwc{axes} \hlstd{=} \hlnum{FALSE}\hlstd{)}
\hlcom{# </code r> ================================================================= #}
\end{alltt}
\end{kframe}

{\centering \includegraphics[width=\maxwidth]{iBagens/unnamed-chunk-2-1} 

}



\end{knitrout}

\textbf{Voc� tamb�m deve implementar os seguintes classificadores, assim como
        estimar seus riscos via conjunto de teste (usando o conjunto de
        valida��o para estimar seus erros), mostrar o resultado de cada um:} \\

\textbf{c) Bagging}

\horrule{.5pt} \\

\textbf{d) �rvore de Classifica��o}

\horrule{.5pt} \\

\textbf{e) Boosting}

\horrule{.5pt} \\

\textbf{f) SVM}

\horrule{.5pt} \\

\textbf{Escolha um dos classificadores ajustados (o que achar melhor) e submeta
        suas respostas ao site}

\vspace{\fill}

\horrule{1pt} \\

\end{document}
